\documentclass[12pt]{article}

\input preamble

\title{Principles of Parallel Architecture\\
Homework 2: Parallel Programming Models}
\author{Xitong Liu \\
xliu@ece.udel.edu}

\begin{document}

\maketitle

\section{MPI Program development}
\begin{enumerate}

\item Ring program
\begin{description}
\item[Q:] The MPI ring program is one of the simplest programs 
that can be written in MPI. The intention of a MPI ring 
program is to write, in order, numbers from 1 to p, where 
p is the total number of processes being used. As a further 
restriction, each process can only write a single, unique 
number in the following way:
\begin{verbatim}
Process number 1 writes number 1
Process number 2 writes number 2
...
Process number p writes number p
\end{verbatim}
\item[A:]  The source code can be located both in \texttt{ring1} 
and \texttt{ring2} directories, with \texttt{Makefile} provided.
Type \texttt{make run} to compile and run the program with results
generated.
\end{description}

\item
\begin{description}
\item[Q:] The parallel sum program: The objective of this 
program is to obtain the summation of 1000 numbers using 
N processors. To do so, at the beginning the numbers are 
distributed among the available processes.
\end{description}
\item[A:] The source code can be located in \texttt{mpi\_parallel\_sum}.
Type \texttt{make run} to compile and run the program with results
generated.
\end{enumerate}

\section{OpenMP program development}
\begin{enumerate}

\item Parallel hello world program
\begin{description}
\item[Q:] This program will print �Hello world from thread i of p� 
where i is the thread number and p is the total number of threads. 
The number of threads to be used should be a parameter that is 
passed through the command line.
\item[A:] The source code can be located in \texttt{omp\_hello\_world}.
Type \texttt{make run} to compile and run the program with results
generated.
\end{description}

\item The parallel sum program
\begin{description}
\item[Q:] The program�s intention is the same as before. The number 
of threads to be used should be a parameter that is passed through 
the command line.

However, the partition is not done by the programmer; instead, use 
the parallel for clause of openmp to distribute numbers to threads 
and to initialize the random numbers.

Use a private variable to accumulate the result in the loop and 
then use atomic operations to produce the final sum.

Measure the running time of the parallel computation and make a 
graph time vs number of threads. Comment about the results.

\item[A:] The source code can be located in \texttt{omp\_parallel\_sum}.
Type \texttt{make run} to compile and run the program with results
generated. The time vs number of threads was plotted in Fig.\ref{fig:omp_sum}.
It's clear to find out that the running time reached mininum value when the 
number of threads is 8, the same as the number of CPUs on the pacific machine.
It's mainly because 8 threads can run on 8 cores simultaneously without context
switch and reach maximum performance.
\begin{figure}[h!]
	\begin{center}
		\includegraphics[width=0.9\textwidth, angle=0]{omp_sum.pdf}
		\caption{\label{fig:omp_sum}OpemMP Sum time vs number of threads}
	\end{center}
\end{figure}
\end{description}
\end{enumerate}

\section{Pthread program development}
\begin{enumerate}

\item The parallel hello world program
\begin{description}
\item[Q:] Achieve the same results obtained using OpenMP, but this time
use pthreads.
\item[A:] The source code can be located in \texttt{pthread\_hello\_world}.
Type \texttt{make run} to compile and run the program with results
generated.
\end{description}

\item The parallel sum program
\begin{description}
\item[Q:] The objective is the same as the previous parallel sum programs. 
However, thread synchronization is not explicit in pthread programming.

Measure the running time of the parallel computation and make a graph 
time vs number of threads. Comment about the results.
\item[A:] The source code can be located in \texttt{pthread\_parallel\_sum}.
Type \texttt{make run} to compile and run the program with results
generated. The time vs number of threads was plotted in Fig.\ref{fig:pthread_sum}.
It's clear to find out that the running time reached mininum value when the 
number of threads is 8, the same as number of CPUs on the pacific machine.
It's mainly because 8 threads can run on 8 cores simultaneously without context
switch and reach maximum performance.
\begin{figure}[h!]
	\begin{center}
		\includegraphics[width=1.0\textwidth, angle=0]{pthread_sum.pdf}
		\caption{\label{fig:pthread_sum}Pthread Sum time vs number of threads}
	\end{center}
\end{figure}
\end{description}
\end{enumerate}

\section{Cilk programming}
\begin{enumerate}

\item The Head-Tails game
\begin{description}
\item[Q:] In the head-tails game, a player throws a coin 5 times and 
then he counts the number of times that he obtained heads. If the player 
obtained heads at least 3 times, he wins, otherwise he loses.
\item[A:] The source code can be located in \texttt{cilk\_coin\_game}.
Type \texttt{make run} to compile and run the program with results
generated.
\end{description}

\end{enumerate}

\end{document}

\begin{comment}
\begin{figure}[h!]
	\begin{center}
		\includegraphics[width=0.7\textwidth, angle=0]{fatest.png}
		\caption{\label{fig:fatest}Fatest SuperComputer in the world}
	\end{center}
\end{figure}
\end{comment}